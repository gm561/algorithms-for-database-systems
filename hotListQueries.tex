\section{Hot list queries}
Gibbons and Matias consider application of concise and counting
samples in the context of choosing $k$ most frequently occurring
values which known as \textit{hot list queries}. Formally it might be
understood as finding $k$ $\langle value, count \rangle$ pairs for
which $count$ is the largest.

\subsection{Algorithms}
Gibbons and Matias \cite{GM98} compared four algorithm for creating
\textit{hot list queries}. The assumption is that we have a relation
$R$ with $n$ tuples and the footprint bound is $m$, where $m \geq 2k$.
%TODO describe confidential threshold $\sigma$

\subsubsection{Traditional samples}
This approach base on Vitter's reservoir sampling algorithm
\cite{Vit85}.
%TODO describe algorithms of Vitter's more preciesly
%actually there are four of them it is not mentioned which one is
%chosen.
All pairs with $max(c_k, \sigma)$ are reported. Then the counts are
scaled by $\frac{n}{m}$, where $\sigma$ is a confidential threshold.
%TODO this is really unclear what is scaled ? the counts ? before
%after choosing it ? and why they are scaled?

\subsubsection{Concise samples}
For maintaining a sample the \textbf{Algorithm
  \ref{alg:offline-concise-algorithm}} is used. Then the $k$'th
largest $count$ $c_k$ is found. All pairs that $count$ is at least are
treated as \textit{hot queries}.
%TODO more precise description
\subsubsection{Counting samples}
For creating a sample the \textbf{Algorithm
  \ref{alg:maintenace-counting-samples}}. For obtaining the
\textit{hot list queries} we use the same algorithm as for concise
samples.

\subsubsection{Full histogram on disk}
This algorithm is used as a baseline for the algorithms presented
above. It build a full histogram which means that all possible
$\langle value, count \rangle$ pairs are created and stored. Then the
set of the $k$ most frequently occurring is chosen. The remaining are
still stored and used in presence of update. As the working set of
this algorithm might be of $\O(n)$ it is not considered to be
very useful in real applications. However, it provides precise results
and can be used as a baseline for other methods.
