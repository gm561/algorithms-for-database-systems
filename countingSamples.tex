\section{Counting Samples}
Counting samples is another approach presented by Gibbons and Matias
\cite{GM98}. The idea is similar to the one behind the concise
samples. The definition of the $count$ is refined. Namely, the counts
are used to track all occurrences of a value inserted into the
relation since the value was selected for the sample. The counting
samples might be understood as the random experiment in which for each
attribute value that occurs $c$ times in relation $R$ the biased coin
with probability of heads $\frac{1}{\tau}$ is flipped till the first
head. If we denote by $i$ the number of tosses till the first head
then the $count$ equals $c - i + 1$. If there is no head in $c$ throws
then the value does not show up in the sample. Similarly as for the
concise sample the value for which count is greater than one is
represented as a pair $\langle value, count \rangle$. Values for
which count is exactly one are represented by singleton
$value$. Clearly, the values for which count is zero are not
represented in the sample. The counting samples are not uniformly
distributed but it is possible to obtain concise sample from counting
samples. To do that, it is enough to toss a biased coin with head
probability $\frac{1}{\tau}$ $count - 1$ times. The concise sample is
the counting sample for which the $coutn$ is reduced by the number of
tails in the above random experiment. Obviously, if the $count$ is
reduced to one, sample is represented by a singleton. Similarly, if the
$count$ is reduced to zero the sample is not represented.

\subsection{Incremental algorithm for maintenance of counting samples}
Similarly as it was in the  case of concise samples Gibbons and Matias
propose an algorithm for maintaining samples in presence of inserts to
data warehouse. The algorithm was briefly presented on the following
code excerpt

\begin{algorithm}[H]
  \caption{Incremental maintenance of counting samples}
  \label{alg:maintenace-counting-algorithm}
  \textbf{Input:} Tuple $t$ to be inserted
  \begin{algorithmic}
  \Function{InsertTuple}{$t$}
  \If{$t.A$ is a pair $\langle value, count \rangle$ in $S$}
  \State Increment by one $count$ in $\langle value, count \rangle$
  \ElsIf{$t.A$ is a singleton in $S$}
  \State Remove singleton $value$ corresponding to $t.A$ from $S$
  \State Insert $\langle t.A, 2 \rangle$ into $S$
  \ElsIf{$t.A$ is not present in $S$}
  \State {Add $t.A$ as singleton value with propbability $\frac{1}{\tau}$}
  \EndIf
  \If{$footprint(S) > m$}
  \State {MakeSpace()}
  \EndIf
  \EndFunction
  \State{}
  %MakeSpace function
  \Function{MakeSpace}{}
  \State{Raise threshold to $\tau'$}
  \ForAll{s in $S$}
  \State{$count' = 0$}
  \State {Flip a coin with probability of head $\frac{\tau}{\tau'}$}
  \While{Tail is flipped}
  \State{$count' \gets count' + 1$}
  \EndWhile
  \State{$s.count \gets s.count - count'$ }
  \If{$s.count = 0$}
  \State{Remove $s$ from $S$}
  \ElsIf{$c.count = 1$ and $s$ is a pair}
  \State{Remove $s$ from $S$}
  \State{Insert singleton corresponding to $s$ into $S$}
  \Else
  \State{Update the pair corresponding to $s$ with new $count$ value}
  \EndIf
  \EndFor
  \EndFunction
\end{algorithmic}

\end{algorithm}
As one can see this algorithm is similar to the one already described
for concise samples. Careful reader might observed that we did not
provided any solution in case of delation of tuples form data
warehouse for concise samples. Actually this is the motivation for
introducing the counting samples. Delating concise samples in hard. If
we omit delation, it is possible that there exists a representation of
a tuple in the sample but this tuple is no longer present in the data
warehouse. On contrary if we update/remove sample on each delation,
the sample representation is not longer a distribution that
corresponds to the one of data stored in data warehouse. 