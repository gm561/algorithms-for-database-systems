\section{Counting Samples}
Counting samples is another approach presented by Gibbons and Matias
\cite{GM98}. The idea is similar to the one behind the concise
samples. The definition of the $count$ is refined. Namely, the counts
are used to track all occurrences of a value inserted into the
relation since the value was selected for the sample. The counting
samples might be understood as the random experiment in which for each
attribute value that occurs $c$ times in relation $R$ the biased coin
with probability of heads $\frac{1}{\tau}$ is flipped till the first
head. If we denote by $i$ the number of tosses till the first head
then the $count$ equals $c - i + 1$. If there is no head in $c$ throws
then the value does not show up in the sample. Similarly as for the
concise sample the value for which count is greater than one is
represented as a pair $\langle value, count \rangle$. Values for
which count is exactly one are represented by singleton
$value$. Clearly, the values for which count is zero are not
represented in the sample. The counting samples are not uniformly
distributed but it is possible to obtain concise sample from counting
samples. To do that, the following algorithm might be used