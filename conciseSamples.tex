\section{Concise Samples}
The goal of the algorithm that maintain set of \textit{concise
  samples} is to obtain a uniform random sample set of $R.A$ values.
The concise sample might be represented by two structures. It might be
either a pair $\langle value, count \rangle $ or a singleton value. The first
type is used for values that occurs in the sample set more than once
(i.e $count > 1$). The values that are present only once in the sample
set are represented just by $value$. The motivation for this
approach is obviously saving space. As a result for sample size may be
much larger than memory footprint of concise sample. We have that if 
\begin{align*}
  S = {\langle v_1, c_1 \rangle, \dots, \langle v_j, c_j \rangle,
    v_{j+1}, \dots, v_l }
\end{align*}
is a concise sample where values between $1$ and $j$ are $\langle
value, count \rangle$ pairs and values $v_{j+1}, \dots, v_l$ are
singletons, then the we define $sample-size(S)$ as
\begin{align*}
sample-size(S) = l - j + \sum_{i=1}^j c_i
\end{align*}
and memory $footprint(S)$ as 
\begin{align*}
footprint(S) = l + j.
\end{align*}

The approach presented by Gibbons and Matias focuses mostly on random
sample sets of a single attribute of some relation
i.e. \textit{R.A}. They claim that it is possible to generalize their
approach for samples sets with pair of attributes. However, it
leads to much bigger values space i.e $ |R.A0| \times |R.A1|$
where by $|R.Ai|$ we denote the number of possible values taken by
attribute $R.Ai$. A concise sample is treated as a uniform sample of
size $sample-size(S)$ and might be used in any sample-base methods for
providing approximate query answers.

\subsection{Offline/static computation concise samplesalgorithm}
Gibbons and Matias \cite{GM98} proposed the an algorithm that extracts
concise samples from the static relation residing on a disk.
By $n$ we denote the number of tuples in the relation, $m$ is the
maximum footprint and $S$ is a concise sample set.

\begin{algorithm}
  \caption{Offline/static concise sample computation}
  \label{alg:offline-concise-algorithm}
\begin{algorithmic}
  \State $m \gets$ maximum footprint
  \State $n \gets$ number of tuples in the relations
  \State $S \gets  \varnothing$
  \While{$footprint(S) \leq m $ or $sample-size(S) \leq n$}
    \State $T \gets$ randomly chose tuple from the relation $R$
    \State $value(A) \gets$ extract attribute $R.A$ from tuple $T$
    \If {$value(A)$ is represented by singleton value in $S$}
        \State remove singleton $value(A)$ from S.
        \State put pair $\langle value(A), 2 \rangle$ to $S$
    \ElsIf {$value(A)$ is represented by a pair in $S$}
        \State increase $count$ in $\langle value(A), count
        \rangle \gets$ by one.
    \Else
        \State add singleton $\angle value(A) \rangle$ to $S$
    \EndIf
    \EndWhile
  \end{algorithmic}
\end{algorithm}

Let $m'$ be the sample-size of $S$. Then the above algorithm requires
$\Theta(m')$ number of disk accesses. This algorithm might be seen
as drawing with replacement (i.e. the chosen tuples are
not removed from the relation). As a result of its execution we obtain
an approximation of tuples distribution in the relation.

\subsection{Incremental maintenance of concise samples}
The algorithm \ref{alg:offline-concise-algorithm} requires disk
accesses. As a result updating the structures is slow. That is ways
the \cite{GM98} presented another algorithm that requires no disk
accesses and let maintain sampling frequency of concise samples that
reflects distribution of tuples in the relation in data base system.
The novelty of Gibbons and Matias approach in comparison to previous
works follows form the fact that this algorithm do not assume that
the sample-size is know up-front (i.e. it differs from the footprint).
%TODO maybe describe previous approach ?
The sample-size depends not only on the footprint but also on data
distribution. As data distribution in general can not be forseen, we
do not know how many different samples we are going to actually store.

\begin{algorithm}
  \caption{Offline/static concise sample computation}
  \label{alg:maintenance-concise-algorithm}
  \begin{algorithmic}
    \LineComment{$\tau$ - entry threshold}
    \LineComment{$t$ new added tuple}
    \LineComment{$S$ current concise sample}
    \Function{AddNewTuple}{$\tau$, $S$, $t$}
    \State {$isAdded \gets$ true  with probability $\frac{1}{\tau}$,
      false otherwise}
    \If{$isAdded = true$ and $t$ not present in $S$}
    \State {create a singleton value $t.A$ and add it to $S$ }
    \ElsIf{$isAdded = true$ and $t.A$ is a singleton in $S$}
    \State {remove singleton $t.A$ from $S$}
    \State {add $\langle t.A, 2 \rangle$ to $S$}
    \ElsIf{$isAdded = true$ and $t$ present in $S$}
    \State {increase $count$ by one in the pair $\langle t.A, count \rangle$}
    \EndIf
    \EndFunction
  \end{algorithmic}

  \caption{Offline/static concise sample computation}
  \label{alg:make-space-concise-algorithm}
  \begin{algorithmic}
    \Function{MakeSpace}{$\tau$}
    \State {Raise the threshold to some $\tau'$}
    \ForAll{$s$ in $S$}
    \If{$s$ is a singleton value in $S$}
    \State {remove s from $S$ with probability
      $\frac{\tau'}{\tau}$}
    \ElsIf{$s$ is a pair $\langle value, count \rangle$ in $S$}
    \State{$count' \gets 0$ }
    \For{$i = 1 \to count$}
    \State{increase $count'$ with probability $\frac{\tau'}{\tau}$}
    \EndFor
    \EndIf
    \EndFor
    \EndFunction
  \end{algorithmic}
\end{algorithm}


