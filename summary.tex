
% \documentclass[11commpt]{article}

%% We use the memoir class because it offers a many easy to use features.
\documentclass[10pt,a4paper,article,oneside]{memoir}

\counterwithout{section}{chapter}
\usepackage[margin=1in]{geometry}

%% Packages
%% ========

%% LaTeX Font encoding -- DO NOT CHANGE
%\usepackage[OT1]{fontenc}

%% Babel provides support for languages.  'english' uses British
%% English hyphenation and text snippets like "Figure" and
%% "Theorem". Use the option 'ngerman' if your document is in German.
%% Use 'american' for American English.  Note that if you change this,
%% the next LaTeX run may show spurious errors.  Simply run it again.
%% If they persist, remove the .aux file and try again.
\usepackage[english]{babel}

%% Input encoding 'utf8'. In some cases you might need 'utf8x' for
%% extra symbols. Not all editors, especially on Windows, are UTF-8
%% capable, so you may want to use 'latin1' instead.
\usepackage[utf8]{inputenc}

%% This changes default fonts for both text and math mode to use Herman Zapfs
%% excellent Palatino font.  Do not change this.
%\usepackage[sc]{mathpazo}

%% The AMS-LaTeX extensions for mathematical typesetting.  Do not
%% remove.
\usepackage{amsmath,amssymb,amsfonts}

%% NTheorem is a reimplementation of the AMS Theorem package. This
%% will allow us to typeset theorems like examples, proofs and
%% similar.  Do not remove.
%% NOTE: Must be loaded AFTER amsmath, or the \qed placement will
%% break
\usepackage[amsmath,thmmarks]{ntheorem}

%% LaTeX' own graphics handling
\usepackage{graphicx}

%% We unfortunately need this for the Rules chapter.  Remove it
%% afterwards; or at least NEVER use its underlining features.
\usepackage{soul}

%% Some more packages that you may want to use.  Have a look at the
%% file, and consult the package docs for each.
%% See the TeXed file for more explanations

%% [OPT] Multi-rowed cells in tabulars
%\usepackage{multirow}

%% [REC] Intelligent cross reference package. This allows for nice
%% combined references that include the reference and a hint to where
%% to look for it.
\usepackage{varioref}

%% [OPT] Easily changeable quotes with \enquote{Text}
%\usepackage[german=swiss]{csquotes}

%% [REC] Format dates and time depending on locale
\usepackage{datetime}

%% [OPT] Provides a \cancel{} command to stroke through mathematics.
%\usepackage{cancel}

%% [NEED] This allows for additional typesetting tools in mathmode.
%% See its excellent documentation.
\usepackage{mathtools}

%% [ADV] Conditional commands
%\usepackage{ifthen}

%% [OPT] Manual large braces or other delimiters.
%\usepackage{bigdelim, bigstrut}

%% [REC] Alternate vector arrows. Use the command \vv{} to get scaled
%% vector arrows.
\usepackage[h]{esvect}

%% [NEED] Some extensions to tabulars and array environments.
\usepackage{array}

%% [OPT] Postscript support via pstricks graphics package. Very
%% diverse applications.
%\usepackage{pstricks,pst-all}

%% [?] This seems to allow us to define some additional counters.
%\usepackage{etex}

%% [ADV] XY-Pic to typeset some matrix-style graphics
%\usepackage[all]{xy}

%% [OPT] This is needed to generate an index at the end of the
%% document.
%\usepackage{makeidx}

%% [OPT] Fancy package for source code listings.  The template text
%% needs it for some LaTeX snippets; remove/adapt the \lstset when you
%% remove the template content.
\usepackage{listings}
\lstset{language=TeX,basicstyle={\normalfont\ttfamily}}

%% [REC] Fancy character protrusion.  Must be loaded after all fonts.
\usepackage[activate]{pdfcprot}

%% [REC] Nicer tables.  Read the excellent documentation.
\usepackage{booktabs}

%%pseudocode and algorithms
\usepackage{algpseudocode}
\usepackage{algorithm}

\algnewcommand{\LineComment}[1]{\State \(\triangleright\) #1}


%% Our layout configuration.  DO NOT CHANGE.
%%% Memoir layout setup

%% NOTE: You are strongly advised not to change any of them unless you
%% know what you are doing.  These settings strongly interact in the
%% final look of the document.

% Dependencies
% Turn extra space before chapter headings off.
\setlength{\beforechapskip}{0pt}

\nonzeroparskip
\parindent=0pt
\defaultlists

% Chapter style redefinition
\makeatletter

\if@twoside
  \pagestyle{Ruled}
  \copypagestyle{chapter}{Ruled}
\else
  \pagestyle{ruled}
  \copypagestyle{chapter}{ruled}
\fi
\makeoddhead{chapter}{}{}{}
\makeevenhead{chapter}{}{}{}
\makeheadrule{chapter}{\textwidth}{0pt}
\copypagestyle{abstract}{empty}

\makechapterstyle{bianchimod}{%
  \chapterstyle{default}
  \renewcommand*{\chapnamefont}{\normalfont\Large\sffamily}
  \renewcommand*{\chapnumfont}{\normalfont\Large\sffamily}
  \renewcommand*{\printchaptername}{%
    \chapnamefont\centering\@chapapp}
  \renewcommand*{\printchapternum}{\chapnumfont {\thechapter}}
  \renewcommand*{\chaptitlefont}{\normalfont\huge\sffamily}
  \renewcommand*{\printchaptertitle}[1]{%
    \hrule\vskip\onelineskip \centering \chaptitlefont\textbf{\vphantom{gyM}##1}\par}
  \renewcommand*{\afterchaptertitle}{\vskip\onelineskip \hrule\vskip
    \afterchapskip}
  \renewcommand*{\printchapternonum}{%
    \vphantom{\chapnumfont {9}}\afterchapternum}}

% Use the newly defined style
\chapterstyle{bianchimod}

\setsecheadstyle{\Large\bfseries\sffamily}
\setsubsecheadstyle{\large\bfseries\sffamily}
\setsubsubsecheadstyle{\bfseries\sffamily}
\setparaheadstyle{\normalsize\bfseries\sffamily}
\setsubparaheadstyle{\normalsize\itshape\sffamily}
\setsubparaindent{0pt}

% Set captions to a more separated style for clearness
\captionnamefont{\sffamily\bfseries\footnotesize}
\captiontitlefont{\sffamily\footnotesize}
\setlength{\intextsep}{16pt}
\setlength{\belowcaptionskip}{1pt}

% Set section and TOC numbering depth to subsection
\setsecnumdepth{subsection}
\settocdepth{subsection}

%% Titlepage adjustments
\pretitle{\vspace{0pt plus 0.7fill}\begin{center}\HUGE\sffamily\bfseries}
\posttitle{\end{center}\par}
\preauthor{\par\begin{center}\let\and\\\Large\sffamily}
\postauthor{\end{center}}
\predate{\par\begin{center}\Large\sffamily}
\postdate{\end{center}}

\def\@advisors{}
\newcommand{\advisors}[1]{\def\@advisors{#1}}
\def\@department{}
\newcommand{\department}[1]{\def\@department{#1}}
\def\@thesistype{}
\newcommand{\thesistype}[1]{\def\@thesistype{#1}}

\renewcommand{\maketitlehooka}{\noindent\ETHlogo[2in]}

\renewcommand{\maketitlehookb}{\vspace{1in}%
  \par\begin{center}\Large\sffamily\@thesistype\end{center}}

\renewcommand{\maketitlehookd}{%
  \vfill\par
  \begin{flushright}
    \sffamily
    \@advisors\par
    \@department, ETH Z\"urich
  \end{flushright}
}

\checkandfixthelayout

\setlength{\droptitle}{-48pt}

\makeatother

% This defines how theorems should look. Best leave as is.
\theoremstyle{plain}
\setlength\theorempostskipamount{0pt}

%%% Local Variables:
%%% mode: latex
%%% TeX-master: "thesis"
%%% End:


%% Theorem environments.
%% thesis.
%% Theorem-like environments

%% This can be changed according to language. You can comment out the ones you
%% don't need.

\numberwithin{equation}{chapter}

%% German theorems
%\newtheorem{satz}{Satz}[chapter]
%\newtheorem{beispiel}[satz]{Beispiel}
%\newtheorem{bemerkung}[satz]{Bemerkung}
%\newtheorem{korrolar}[satz]{Korrolar}
%\newtheorem{definition}[satz]{Definition}
%\newtheorem{lemma}[satz]{Lemma}
%\newtheorem{proposition}[satz]{Proposition}

%% English variants
\newtheorem{theorem}{Theorem}[section]
\newtheorem{example}[theorem]{Example}
\newtheorem{remark}[theorem]{Remark}
\newtheorem{corollary}[theorem]{Corollary}
\newtheorem{definition}[theorem]{Definition}
\newtheorem{lemma}[theorem]{Lemma}
\newtheorem{proposition}[theorem]{Proposition}

%% Proof environment with a small square as a "qed" symbol
\theoremstyle{nonumberplain}
\theorembodyfont{\normalfont}
\theoremsymbol{\ensuremath{\square}}
\newtheorem{proof}{Proof}
%\newtheorem{beweis}{Beweis}



%% Helpful macros.
%% Custom commands
%% ===============

%% Special characters for number sets, e.g. real or complex numbers.
\newcommand{\C}{\mathbb{C}}
\newcommand{\K}{\mathbb{K}}
\newcommand{\N}{\mathbb{N}}
\newcommand{\Q}{\mathbb{Q}}
\newcommand{\R}{\mathbb{R}}
\newcommand{\Z}{\mathbb{Z}}
\newcommand{\X}{\mathbb{X}}

%% Fixed/scaling delimiter examples (see mathtools documentation)
\DeclarePairedDelimiter\abs{\lvert}{\rvert}
\DeclarePairedDelimiter\norm{\lVert}{\rVert}

%% Use the alternative epsilon per default and define the old one as \oldepsilon
\let\oldepsilon\epsilon
\renewcommand{\epsilon}{\ensuremath\varepsilon}

%% Also set the alternate phi as default.
\let\oldphi\phi
\renewcommand{\phi}{\ensuremath{\varphi}}


\usepackage{enumerate}

\begin{document}

\title{Sampling based summary statistics}
\author{Grzegorz Makosa}
\maketitle

%present what is included in this paper
%TODO add what I am going to write about
%TODO approaches?
\section{Abstract}
This paper summarizes different methods for sampling based summary
statistics discussed by Gibbons and Matias
\cite{GM98}. We present new structures developed by
those authors - concise samples and counting samples, analyze algorithms to
maintain them in presence of insertion and delation of tuples
stored in data warehouse.  Finally, we show a practical application
of concise and counting samples to provide fast approximate answers
to hot list queries.



%quick introduction to the topic of sampling based summary statistics
\section{Introduction}
The increase in data size stored in data warehouses and simultaneous
requirement to obtain query answers within a reasonable time creates a
need to develop new methods that provide fast even only approximated
results. Techniques that base entirely on direct access to a data
warehouse appear to be not efficient enough due to numerous
operation involving slow disk. There are some applications
for which obtaining results quickly, even if they are only
approximation, is far better than waiting for slow but exact
answer. Therefore, fast and approximated methods
constitute a considerable alternative for accessing data stored in a
data warehouse.

Gibbons and Matias \cite{GM98} proposed techniques that attempt to meet a
need of users that require fast, approximated answers. Their solution use not
only data warehouse but also an approximate answer engine.
The approximate answer engine is responsible for storing
sample based summary statistics, that are called \textit{synopsis data
  structures} or \textit{synopses}. Instead
directly to a data warehouse, queries are sent to approximate answer
engine which using \textit{synopses} returns approximate results
together with accuracy measure.
The users can estimate usefulness of the query answer and depending on it
decide whether to query proper database in order to obtain exact results or not.
The advantage of using approximate query engine is smaller working set.
Therefore, it can use the faster high level memory to return query answers
which results in immediate response.

The novelty of techniques developed by Gibbons and Matias
\cite{GM98} base on models providing samples
that better reflect distribution of
data stored in a data warehouses and simultaneously maintain small working set.
This techniques differ from the previous ones in which either the
entire data warehouse is scanned and the approximate answer is being updated as
the scan proceeds or in which data sample is created in the less
memory efficient way.

The crucial observation is the fact that techniques that provide
fast response to various type of queries might be developed if the
frequently used structures are stored in high level memory
(i.e. memory close to processor like caches or main memory). From this follows
that it is beneficial to find a way to limit size of used data structures. Therefore,
it is useful to consider the effectiveness of used data structures
as a function of memory footprint and the number of tuples represented
in the sample. Additionally, the measurement of effectiveness
should take into account two factors - the
accuracy of the provided answers and the response time.

Main difficulty related to data structures providing approximated answers is
keeping them up to date, so that they reflects correctly the content
of the data warehouse even in presence of changes.
Gibbons and Matias \cite{GM98} proposed algorithms that let
maintain synopses data structures in presence of insertion
or deletion of tuples.


\subsection{Concise and counting samples}
Gibbons and Matias \cite{GM98} introduced two new sampling based summary
statistics - concise samples and counting
  samples. However the basic idea behind this methods is simple, it
also requires development of algorithms for fast incremental
maintenance of this structures in face of frequent updates of data
warehouse.

The concise samples and counting samples focus on the class of queries
that ask for a one or more attributes of a single data relation.

The previous solutions like the one proposed by Vitter \cite{Vit85}
base on selecting a random set of tuples from the relation.
This sample set establish an approximation of distribution of values
in a relation. The quality of the representation depends on the size of
the sample. Gibbons and Matias observed that the frequently occurring values in the
sample lead to inefficient use of space and hence, it representation
is not optimal.
They proposed an alternative way to represent frequently occurring
value in the sample as a pair $\langle value, count \rangle$ where
$value$ is a value of an attribute in a relation and $count$ is the
number of this value copies in the sample. If the attribute value is
represented by a type that size is equal to an integer numeric
type the same that is used to represent $count$, then a single pair
save space for $count - 2$ additional points.

This idea is a base for concise sample data structure. Formally,
concise sample is defined as a uniform
random sample of the data set such that values appearing more than
once in the sample are represented as a value and a count pair which
is denoted further as $\langle value, count \rangle$. If the
value occurs exactly once in the sample it is represented as a
singleton value, is denote by $value$ and its size is the same as a
single the size of a single data point.

% Gibbons and
% Matias argues that this simple method is powerful as
% freeing up space let move data to the memory that is closer to the
% processor which substantially reduce access time. What is more, this
% technique is never worse (in the sense of occupied memory) that
% rudimentary random samples. The method used to represent concise
% samples might be also seen as a way to improve the quality of
% approximation answer, as the saved memory might be used for additional
% samples. Hence, the result is more accurate.

The second technique proposed in the paper \cite{GM98} are
counting samples. They are conceptually similar to concise
samples as they also make use of representation of frequently
occurring values in the samples as $\langle value, count \rangle$ but
provides more flexibility. In particular counting samples work also in
presence of delation of tuples from the data warehouse which is not
possible when concise samples are used.

\subsection{Previous techniques}
Already before 1998 when Gibbons and Matias published their paper
there had been known some techniques for obtaining
approximate query answers. One of them was online aggregation
proposed by Hellerstein, Hass and Wang \cite{HHW97}. This
framework based technique relaying on scanning database on random.
The approximate answer is being developed
and displayed to the user together with confidence interval. It is
possible to stop the scan in any time. This technique requires access
to disk memory and algorithms to access tuples of
relation randomly, which makes it less efficient.

Another approach proposed by Vitter \cite{Vit85} is
similar to concise and counting samples. It creates a sample of tuples
of a relation. Nevertheless, it less memory efficient as each value is
represented independently and does not provide accumulated pairs as
in the methods developed by Gibbons and Matias.



%concise samples
\section{Concise Samples}
The goal of the algorithm that maintain set of \textit{concise
  samples} is to obtain a uniform random sample set of $R.A$ values.
The concise sample might be represented by two structures. It might be
either a pair $\langle value, count \rangle $ or a singleton value. The first
type is used for values that occurs in the sample set more than once
(i.e $count > 1$). The values that are present only once in the sample
set are represented just by $value$. The motivation for this
approach is obviously saving space. As a result for sample size may be
much larger than memory footprint of concise sample. We have that if 
\begin{align*}
  S = {\langle v_1, c_1 \rangle, \dots, \langle v_j, c_j \rangle,
    v_{j+1}, \dots, v_l }
\end{align*}
is a concise sample where values between $1$ and $j$ are $\langle
value, count \rangle$ pairs and values $v_{j+1}, \dots, v_l$ are
singletons, then the we define $sample-size(S)$ as
\begin{align*}
sample-size(S) = l - j + \sum_{i=1}^j c_i
\end{align*}
and memory $footprint(S)$ as 
\begin{align*}
footprint(S) = l + j.
\end{align*}

The approach presented by Gibbons and Matias focuses mostly on random
sample sets of a single attribute of some relation
i.e. \textit{R.A}. They claim that it is possible to generalize their
approach for samples sets with pair of attributes. However, it
leads to much bigger values space i.e $ |R.A0| \times |R.A1|$
where by $|R.Ai|$ we denote the number of possible values taken by
attribute $R.Ai$. A concise sample is treated as a uniform sample of
size $sample-size(S)$ and might be used in any sample-base methods for
providing approximate query answers.

\subsection{Offline/static computation concise samplesalgorithm}
Gibbons and Matias \cite{GM98} proposed the an algorithm that extracts
concise samples from the static relation residing on a disk.
By $n$ we denote the number of tuples in the relation, $m$ is the
maximum footprint and $S$ is a concise sample set.

\begin{center}
  \captionof{algorithm}{Offline/static concise sample computation}
    \label{alg:offline-concise-algorithm}
\begin{algorithmic}
  \State $m \gets$ maximum footprint
  \State $n \gets$ number of tuples in the relations
  \State $S \gets  \varnothing$
  \While{$footprint(S) \leq m $ or $sample-size(S) \leq n$}
    \State $T \gets$ randomly chose tuple from the relation $R$
    \State $value(A) \gets$ extract attribute $R.A$ from tuple $T$
    \If {$value(A)$ is represented by singleton value in $S$}
        \State remove singleton $value(A)$ from S.
        \State put pair $\langle value(A), 2 \rangle$ to $S$
    \ElsIf {$value(A)$ is represented by a pair in $S$}
        \State increase $count$ in $\langle value(A), count
        \rangle \gets$ by one.
    \Else
        \State add singleton $\angle value(A) \rangle$ to $S$
    \EndIf
    \EndWhile
  \end{algorithmic}
\end{center}

Let $m'$ be the sample-size of $S$. Then the above algorithm requires
$\Theta(m')$ number of disk accesses. This algorithm might be seen
as drawing with replacement (i.e. the chosen tuples are
not removed from the relation). As a result of its execution we obtain
an approximation of tuples distribution in the relation.

\subsection{Incremental maintenance of concise samples}
The \textbf{Algorithm \ref{alg:offline-concise-algorithm}} requires
linear number of disk
accesses. As a result updating the structures is slow. That is why
the \cite{GM98} presents an online algorithm that requires no disk
accesses and maintains sampling frequency of concise samples that
reflects distribution of tuples stored in the relation in data base system.
%TODO maybe describe previous approach ?
%TODO read the paper by Vitter and compare this work to it.
The novelty of Gibbons and Matias approach in comparison to previous
works follows from the fact that this algorithm do not assume that
sample-size is know up-front (i.e. it differs from the footprint).
The sample-size depends not only on the footprint but also on data
distribution. As distribution in general is not known at the beginning
and might change in time, it is actually not known how many different
samples will be stored. The following algorithm maintains concise
samples during insertion

\begin{center}
  \captionof{algorithm}{Incremental maintenance of concise samples.}
    \label{alg:maintenance-concise-algorithm}
    \begin{algorithmic}
    \LineComment{$\tau$ - entry threshold}
    \LineComment{$t$ new added tuple}
    \LineComment{$S$ current concise sample}
    \Function{AddNewTuple}{$\tau$, $S$, $t$}
    \State {$isAdded \gets$ true  with probability $\frac{1}{\tau}$,
      false otherwise}
    \If{$isAdded = true$ and $t$ not present in $S$}
    \State {create a singleton value $t.A$ and add it to $S$ }
    \ElsIf{$isAdded = true$ and $t.A$ is a singleton in $S$}
    \State {remove singleton $t.A$ from $S$}
    \State {add $\langle t.A, 2 \rangle$ to $S$}
    \ElsIf{$isAdded = true$ and $t$ present in $S$}
    \State {increase $count$ by one in the pair $\langle t.A, count \rangle$}
    \EndIf
    \If{footprint of $S > m$}
    \State{call MakeSpace()}
    \EndIf
    \EndFunction
    \State
    \Function{MakeSpace}{}
    \State {Raise the threshold to some $\tau'$}
    \ForAll{$s$ in $S$}
    \If{$s$ is a singleton value in $S$}
    \State {remove s from $S$ with probability
      $\frac{\tau'}{\tau}$}
    \ElsIf{$s$ is a pair $\langle value, count \rangle$ in $S$}
    \State{$count' \gets 0$}
    \For{$i = 1 \to count$}
    \State{increase $count'$ with probability $\frac{\tau'}{\tau}$}
    \EndFor
    \State{$count \gets count - count'$}
    \If{$count = 0$}
    \State{remove $\langle value, count \rangle$ from $S$}
    \ElsIf{$count = 1$ and $value$ is represented as a pair}
    \State{Remove pair $\langle value, count \rangle$ from $S$}
    \State{Insert singleton $value$ into $S$}
    \EndIf
    \EndIf
    \EndFor
    \State {$\tau \gets \tau'$}
    \EndFunction
  \end{algorithmic}
\end{center}

%TODO add more detail description of the algorithm.
The key invariant of the \textbf{Algorithm
  \ref{alg:maintenance-concise-algorithm}} is that each tuple is in
represented in concise sample as if the current value of the threshold
$\tau$ were always constant. In other words probability of a tuple being
represented in concise sample is equal $\frac{1}{\tau}$ even if the
$\tau$ has been changing in the past. Gibbons and Matias \cite{GM98} provide proof
of this invariant which seems to be not fully correct. Below this
proof has been sketch.
\begin{theorem}
Each tuple in the relation is treated as the invariant was always
$\tau$ even if it has changed from some $\tau'$ to $\tau$.
\end{theorem}
\begin{proof}
Let $X_{t\tau}$ be an indicator random variable of the event that tuple
t is in the sample when the threshold is $\tau$. Clearly, we have
\begin{align*}
P(X_{t\tau} = 1) = \frac{1}{\tau}
\end{align*}
The probability of the tuple $t$ to be in the sample after changing a
threshold to $\tau'$ equals

\begin{align*}
  P(X_{t\tau'} = 1) &= P(X_{t\tau'} = 1 | X_{t\tau} = 1) P(X_{t\tau} = 1)
  + P(X_{t\tau'} = 1 | X_{t\tau} = 1) P(X_{t\tau} = 1) \\
  &=  P(X_{t\tau'} = 1 | X_{t\tau} = 1) P(X_{t\tau} = 1) \\
  &= \frac{\tau}{\tau'} \frac{1}{\tau} = \frac{1}{\tau'}.
\end{align*}
\end{proof}
Hence, if the current threshold is $\tau$, the probability of a tuple
to be in the sample equals $\frac{1}{\tau}$ . The only problem
with this proof is that the authors in \cite{GM98} define the
probability of evicting a sample to be
\begin{align*}
P(X_{t\tau'} = 0 | X_{t\tau} = 1) = \frac{\tau}{\tau'}.
\end{align*}
It also follows from the
\textbf{Algorithm \ref{alg:maintenance-concise-algorithm}}. I assume
that this is just a slight inconsistency and in fact a tuple after
change of threshold should
remain in the sample given that it is in the sample before the
threshold was changed with probability $\frac{\tau'}{\tau}$.

It is possible that flipping a coin for each insertion of new data can
create certain overhead. Basing on the reservoir sampling Algorithm X
\cite{Vit85} Gibbons and Matias propose an approach in which a coin is
flipped in order to determine how many consecutive inserts will be
omitted. This idea bases on generating a value for a random
variable with give distribution. As it is known that the probability
of skipping over exactly next $i$ elements is
\begin{align*}
P(\text{skip over exactly next $i$ elements}) = \left( 1 - \frac{1}{\tau}\right)^i\frac{1}{\tau}
\end{align*}
%TODO describe algorihtm for skipping over elements described in
%vitter...
%TODO add an example that show exponential advantage of concise
%samples page 334.

\subsection{Experimental evaluation}
Gibbons and Matias \cite{GM98} compared the Algorithms
\ref{alg:offline-concise-algorithm}, \ref{alg:maintenance-concise-algorithm} with approach
when a random sample is maintained using the reservoir
algorithm, in which case the sample size is equal to the footprint.

The experimental results for the Zpif distribution shows that in
general the overhead connected with choosing a threshold is not
critical. Gibbons and Matias \cite{GM98} used simple model in which
the threshold is increase by $10\%$ of the previous one. It turned out
that even using such a simple model the difference between offline
Algorithm \ref{alg:offline-concise-algorithm} and the maintenance
Algorithm \ref{alg:maintenance-concise-algorithm} are not
big. It would be possible to achieve even better results when for
example a binary search would be used to find an optimal
threshold. 

%counting samples
\section{Counting Samples}
Counting samples is another approach presented by Gibbons and Matias
\cite{GM98}. The idea is similar to the one behind the concise
samples. The definition of the $count$ is refined. Namely, the counts
are used to track all occurrences of a value inserted into the
relation since the value was selected for the sample. The counting
samples might be understood as the random experiment in which for each
attribute value that occurs $c$ times in relation $R$ the biased coin
with probability of heads $\frac{1}{\tau}$ is flipped till the first
head. If we denote by $i$ the number of tosses till the first head
then the $count$ equals $c - i + 1$. If there is no head in $c$ throws
then the value does not show up in the sample. Similarly as for the
concise sample the value for which count is greater than one is
represented as a pair $\langle value, count \rangle$. Values for
which count is exactly one are represented by singleton
$value$. Clearly, the values for which count is zero are not
represented in the sample. The counting samples are not uniformly
distributed but it is possible to obtain concise sample from counting
samples. To do that, it is enough to toss a biased coin with head
probability $\frac{1}{\tau}$ $count - 1$ times. The concise sample is
the counting sample for which the $coutn$ is reduced by the number of
tails in the above random experiment. Obviously, if the $count$ is
reduced to one, sample is represented by a singleton. Similarly, if the
$count$ is reduced to zero the sample is not represented.

\subsection{Incremental algorithm for maintenance of counting samples}
Similarly as it was in the  case of concise samples Gibbons and Matias
propose an algorithm for maintaining samples in presence of inserts to
data warehouse. The algorithm was briefly presented on the following
code excerpt

\begin{algorithm}[H]
  \caption{Incremental maintenance of counting samples}
  \label{alg:maintenace-counting-algorithm}
  \textbf{Input:} Tuple $t$ to be inserted
  \begin{algorithmic}
  \Function{InsertTuple}{$t$}
  \If{$t.A$ is a pair $\langle value, count \rangle$ in $S$}
  \State Increment by one $count$ in $\langle value, count \rangle$
  \ElsIf{$t.A$ is a singleton in $S$}
  \State Remove singleton $value$ corresponding to $t.A$ from $S$
  \State Insert $\langle t.A, 2 \rangle$ into $S$
  \ElsIf{$t.A$ is not present in $S$}
  \State {Add $t.A$ as singleton value with propbability $\frac{1}{\tau}$}
  \EndIf
  \If{$footprint(S) > m$}
  \State {MakeSpace()}
  \EndIf
  \EndFunction
  \State{}
  %MakeSpace function
  \Function{MakeSpace}{}
  \State{Raise threshold to $\tau'$}
  \ForAll{s in $S$}
  \State{$count' = 0$}
  \State {Flip a coin with probability of head $\frac{\tau}{\tau'}$}
  \While{Tail is flipped}
  \State{$count' \gets count' + 1$}
  \EndWhile
  \State{$s.count \gets s.count - count'$ }
  \If{$s.count = 0$}
  \State{Remove $s$ from $S$}
  \ElsIf{$c.count = 1$ and $s$ is a pair}
  \State{Remove $s$ from $S$}
  \State{Insert singleton corresponding to $s$ into $S$}
  \Else
  \State{Update the pair corresponding to $s$ with new $count$ value}
  \EndIf
  \EndFor
  \EndFunction
\end{algorithmic}

\end{algorithm}
As one can see this algorithm is similar to the one already described
for concise samples. Careful reader might observed that we did not
provided any solution in case of delation of tuples form data
warehouse for concise samples. Actually this is the motivation for
introducing the counting samples. Delating concise samples in hard. If
we omit delation, it is possible that there exists a representation of
a tuple in the sample but this tuple is no longer present in the data
warehouse. On contrary if we update/remove sample on each delation,
the sample representation is not longer a distribution that
corresponds to the one of data stored in data warehouse. 

%related works - maybe also the newest ones?
\section{Hot list queries}
Gibbons and Matias consider application of concise and counting
samples in the context of choosing $k$ most frequently occurring
values which known as \textit{hot list queries}. Formally it might be
understood as finding $k$ $\langle value, count \rangle$ pairs for
which $count$ is the largest.

\subsection{Algorithms}
Gibbons and Matias \cite{GM98} compared four algorithm for creating
\textit{hot list queries}. The assumption is that we have a relation
$R$ with $n$ tuples and the footprint bound is $m$, where $m \geq 2k$.
%TODO describe confidential threshold $\sigma$

\subsubsection{Traditional samples}
This approach base on Vitter's reservoir sampling algorithm
\cite{Vit85}.
%TODO describe algorithms of Vitter's more preciesly
%actually there are four of them it is not mentioned which one is
%chosen.
All pairs with $max(c_k, \sigma)$ are reported. Then the counts are
scaled by $\frac{n}{m}$, where $\sigma$ is a confidential threshold.
%TODO this is really unclear what is scaled ? the counts ? before
%after choosing it ? and why they are scaled?

\subsubsection{Concise samples}
First we have to find the $k$'th largest count. Let us denote it by
$c_k$. All pairs that $count$ is at least $max(c_k, \sigma)$ are in
the sample. The $count$s are then scaled by $\frac{n}{m'}$ where $m'$
is the sample-size of the concise sample.
%what happens in the case of \sigma = 1.
For maintaining a sample the \textbf{Algorithm
  \ref{alg:offline-concise-algorithm}} is used.

\subsubsection{Counting samples}
The approach for determining approximate hot list is similar to the
one presented above for concise samples. Namely, the $c_k$ is
determined, all pairs that $count$ is at least $max(c_k, \sigma)$ are
selected. What is different is the methods for scaling the counts.

For creating a sample the \textbf{Algorithm
  \ref{alg:maintenace-counting-samples}}.

\subsubsection{Full histogram on disk}
This algorithm is used as a baseline for the algorithms presented
above. It build a full histogram which means that all possible
$\langle value, count \rangle$ pairs are created and stored. Then the
set of the $k$ most frequently occurring is chosen. The remaining are
still stored and used in presence of update. As the working set of
this algorithm might be of $\O(n)$ it is not considered to be
very useful in real applications. However, it provides precise results
and can be used as a baseline for other methods.


%conclusion
\input{conclusion}

\bibliographystyle{alpha}
\bibliography{refs}

\end{document}

